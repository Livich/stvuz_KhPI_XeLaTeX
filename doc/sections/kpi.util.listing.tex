\subsection{Опциональный пакет <<kpi.util.listing>>}
Обеспеичвает поддержку листингов.\par
Макросы:
\begin{enumerate}
\item putListing;
\item listingLink.
\end{enumerate}\par
Внимание: пакет переопределяет настройки пакета listings и переопределяет команды пакета framed.\par
Зависит от пакетов: listings, framed (используется модицифированный пакет из комплекта поставки)
\subsubsection{Макрос <<putListing>>}
putListing вставляет листинг с разбивкой на страницы, как будто это изображения.
Использование:{\small
\begin{Verbatim}
\putListing{[метка]}{[исходный файл]}{[подпись]}{[язык листинга]}
\end{Verbatim}}
\normalsize
При этом будут созданы label <<[метка]-0>>,..., <<[метка]-N>>, <<[метка]-last>>, которые последовательно указывают на первую и последующие страницы листинга. Label <<~-0>> не будет определена, если листинг не разбивался, вместо неё определяется сразу <<~-last>> и это будет единственная метка, указывающая на этот листинг.\par
Также будет создан счётчик <<[метка]:counter>>, содержащий количество страниц в листинге, однако соответствующие действительному числу страниц листинга значения он принимает только после того, как все части листинга были вставлены в документ. Счётчик принимает значения, равные общему числу страниц листинга, включая страницы, не полностью заполненные листингом.

\subsubsection{Макрос <<listingLink>>}
listingLink вставляет ссылку на листинг, учитывая разбивку. Так, если листинг занимает только один рисунок, то будет возвращена ссылка на только один этот рисунок. Если рисунков больше - это тоже учитывается и будут возвращены ссылки на первый и последний рисунки через дефис.
Использование:{\small
\begin{Verbatim}
\listingLink{[метка листинга]}
\end{Verbatim}}
\normalsize
listingLink возвращает значения без какого-либо предваряющего или завершающего пробела.

\subsubsection{Известные проблемы}
Пакет framed, использующийся для обрамления листингов в рамки используется в модифицированном виде. Это связано с тем, что в стандартной имплементации framed не позволяет добавить подпись к объекту в рамке так, чтобы он умещался на той же странице, что и объект в рамке.\par
Пакет framed модифицирован так, что обрамлённый в рамку объект дополняется значением команды theFramedSubCaption, которая определяется в <<kpi.util.listing>>.\par
В настоящий момент проблема с использованием модицифированного пакета не решена и последняя просто включена в комплект поставки набора.

