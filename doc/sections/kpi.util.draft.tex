\subsection{Опциональный пакет <<kpi.util.draft>>}
Обеспеичвает поддержку режима черновика.
Макросы:
\begin{enumerate}
\item ifxdraft;
\item dPutImage;
\item dPutListing;
\item draftPage.
\end{enumerate}
Зависит от пакетов: datetime, ifdraft 

\subsubsection{Макрос <<ifxdraft>>}
Позволяет выполнять проекрку режима черновика.
Использование:{\small
\begin{Verbatim}
\ifxdtaft [если черновик] \else [если не черновик] \fi
\end{Verbatim}}
\normalsize

\subsubsection{Макрос <<dPutImage>>}
Позволяет отобразить вместо изображения его заглушку в режиме черновика, и само изображение в чистовом режиме.
Использование:{\small
\begin{Verbatim}
\dPutImage{[имя файла изображения]}{[подпись]}{[масштаб]}{[метка]}
\end{Verbatim}}
\normalsize
Исползование макроса аналогично putImage.

\subsubsection{Макрос <<dPutListing>>}
Позволяет отобразить вместо листинга его заглушку в режиме черновика, и сам листинг в чистовом режиме.
Использование:{\small
\begin{Verbatim}
\dPutListing{[исходный файл]}{[подпись]}{[метка]}{[язык листинга]}
\end{Verbatim}}
\normalsize
Использование макроса аналогично putListing.

\subsubsection{Макрос <<draftPage>>}
Позволяет подключать к документу-черновику страницу с информацией о версии документа
Использование:{\small
\begin{Verbatim}
\draftPage
\end{Verbatim}}
\normalsize
В результате использования будет отображена страница, извещающая о том, что этот документ является черновиком. Также будет показана таблица со сводной информацией о версии документа.\par
В случае, если документ компилируется в чистовом режиме, команда не делает ничего.