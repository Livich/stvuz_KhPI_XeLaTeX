\section{Введение}
\subsection{Лицензия}
	Данный набор шаблонов выпускается под лицензией CC-BY-SA
	(\url{http://creativecommons.org/licenses/by-sa/3.0/}), то есть никаких
	гарантий на функционал или просто работоспособность не предоставляется. Вы
	можете как угодно изменять этот пакет, с сохранением первоначального авторства
	за его оригинальными авторами (Livich и Vbolshutkin).\par
	Этот набор не является официальным и не поддерживается никаким подразделением
	НТУ <<ХПИ>>. Любые справки могут быть получены только непосредственно в
	репозитории проекта на GitHub:
	\url{https://github.com/Livich/stvuz_KhPI_XeLaTeX}.
\subsection{Принципы использования}
	Этот пакет шаблонов используется так же, как и любой другой пакет для \XeLaTeX,
	однако сам набор поделён на несколько частей: одну обязательную (базовый пакет)
	и множество необязательных, обеспечивающих дополнительный функционал.\par
	В пакете принято соглашение об именовании файлов так, чтобы можно было получить
	доступ к части пакета интуитивно. Базовый пакет пока не подчинён этому
	соглашению.\par
	Таким образом, используя данный набор, необходимо придерживаться следующей
	схемы:
	\begin{enumerate}
	  \item Установить и настроить компилятор \XeLaTeX на вашем компьютере, если
	  это не сделано ранее;
	  \item Установить и настроить Bib\TeX на вашем компьютере, если
	  это не сделано ранее;
	  \item Установить все зависимые пакеты, если
	  это не сделано ранее;
	  \item Распаковать данный набор в доступную для \XeLaTeX-компилятора
	  директорию, если это не сделано ранее;
	  \item Подключить базовый пакет к вашему документу или взять за основу готовый
	  шаблон документа из этого пакета (document.tex);
	  \item Опционально подключить дополнительные пакеты из набора.
	\end{enumerate}
	Именование опциональных пакетов производится следующим образом:
	\begin{itemize}
	  \item Если это пакет-утилита (произвольный дополнительный функционал), то он
	  именуется как {\bf kpi.util.<идентификатор>.sty};
	  \item Если это пакет-титульная страница, то он именуется как {\bf
	  kpi.title.<идентификатор>.sty}.
	\end{itemize}
	Других правил именования пока не предусмотрено.
\subsection{Литература}
	Об установке, настройке и использовании \LaTeX, \XeLaTeX, Bib\TeX написано
	достаточно большое количество материалов, которые могут пригодиться, если вы
	впервые работаете с этой издательской системой.\par
	Полезными окажутся:
	\begin{itemize}
	  \item Книга, С.М. Львовский: <<Набор и вёрстка в системе TEX>>;
	  \item Веб-сайт,
	  \url{http://mydebianblog.blogspot.com/2009/04/miktex-windows.html};
	  \item Веб-сайт, \url{http://www.tug.org/texlive/}.
	\end{itemize}
