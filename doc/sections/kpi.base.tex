\subsection{Общее описание базового пакета}
Базовый пакет содержит наиболее употребимые при оформлении документов команды и настройки. Преимущественно, базовый пакет разработан в рамках проекта \url{https://github.com/vbolshutkin/LaTeX-STVUZ-KhPI} пользователем Vbolshutkin. Однако, в силу того что в оригинальном исполнении этот пакет содержит недостаточно широкий функционал и большое количество известных проблем, он время от времени модифицируется, перерабатывается и дополняется.\par
Базовый пакет предоставляет следующие окружения:
\begin{enumerate}
\item longEnumerate~--~поддержка длинных списков;
\item formulaDescription~--~поддержка вставки описаний для переменных в формулах;
\item abbrDescription~--~поддержка описания аббревиатур;
\item equation~--~поддержка вставки формулы с номером;
\item equation*~--~поддержка вставки формулы без номера;
\item stdtableshort~--~поддержка коротких таблиц;
\item stdtablelong~--~поддержка длинных таблиц.
\end{enumerate}

Также базовый пакет предоставляет следующие макросы:
\begin{enumerate}
  \item startAppendix, appendixSection, appendixSubsection~--~поддержка автоматического оформления приложений;
  \item enquote, enquote*~--~для взятия текста в кавычки.
\end{enumerate}
\subsection{Подключение базового пакета}
Базовый пакет подключается путём указания класса документа:
{\small
\begin{verbatim}
\documentclass{kpi.base}
\end{verbatim}}
\normalsize