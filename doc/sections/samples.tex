\subsection{Опциональный пакет <<kpi.util.image>>}
	Пакет позволяет вставлять изображения.\par
	Пример приведён на рисунке~\ref{putImageExample}.
	\putImage{putImageExample}{glogo.jpg}{Пример вставки изображения}{0.2}	
\subsection{Опциональный пакет <<kpi.util.listing>>}
	Пакет позволяет вставлять листинги в соответствии со стандартом СТВУЗ, то есть оформлять их как рисунки. Листинг разбивается на множество изображений, каждое изображение нумеруется и подписывается.\par
	Пример вставки листинга, автоматических ссылок и автоматической разбивки его на отдельные рисунки приведён на рисунках~\listingLink{kpi-util-listing-first}.
	\putListing{kpi-util-listing-first}{./etc/listing.cpp}{Исходный код файла <<./etc/listing.cpp>>}{c}
\subsection{Опциональный пакет <<kpi.title.coursework>>}
	Пакет позволяет сформировать титульные листы для курсовой работы. Как и любой title-пакет, определяет всего одну команду <<maketitle>>. Для формирования требуется, чтобы были определены дополнительные команды.\par
	Пример сформированных листов приведён в файле <<./etc/title-coursework.pdf>>.
\subsection{Опциональный пакет <<kpi.title.report>>}
	Пакет позволяет сформировать титульный лист типа <<Звіт>> для лабораторных работ. Как и любой title-пакет, определяет всего одну команду <<maketitle>>. Для формирования требуется, чтобы были определены дополнительные команды.\par
	Пример сформированных листов приведён в файле <<./etc/title-report.pdf>>.