\subsection{Опциональный пакет <<kpi.util.image>>}
	Пакет позволяет вставлять изображения.\par
	Пример приведён на рисунке~\ref{putImageExample}.
	\putImage{glogo.jpg}{Пример вставки изображения}{0.2}{putImageExample}	
\subsection{Опциональный пакет <<kpi.util.listing>>}
	Пакет позволяет вставлять листинги в соответствии со стандартом СТВУЗ, то есть оформлять их как рисунки. Листинг разбивается на множество изображений, каждое изображение нумеруется и подписывается.\par
	Пример вставки листинга, автоматических ссылок и автоматической разбивки его на отдельные рисунки приведён на рисунках~\listingLink{kpi-util-listing-first}.
	\putListing{./etc/listing.cpp}{Исходный код файла <<./etc/listing.cpp>>}{kpi-util-listing-first}{c}
\subsection{Опциональный пакет <<kpi.title.coursework>>}
	Пакет позволяет сформировать титульные листы для курсовой работы. Как и любой title-пакет, определяет всего одну команду <<maketitle>>. Для формирования требуется, чтобы были определены следующие команды метаинформации: 
	\begin{enumerate}
	\item docDepartment~--~название кафедры;
	\item workSupervisor~--~руководитель работы, например: <<Сокол В.Е.>>;
	\item workSupervisorPosition~--~статус руководителя работы, например <<асистент каф. АСУ>>;
	\item docSubject~--~название курсовой работы;
	\item studentGroup~--~группа автора;
	\item studentName~--~ФИО автора;
	\item docChecker~--~ФИО проверяющего;
	\item docCheckerPosition~--~статус проверяющего, например <<голова комісії професор>>.	
	\end{enumerate}
	Пример сформированных листов приведён в файле <<./etc/title-coursework.pdf>>.