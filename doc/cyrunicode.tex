%%%%%%%%%% MODERN SLAVIC CYRILLIC LETTERS - COMPLETE
% This definition set has been confirmed as complete for
% all Slavic Cyrillic languages (Russian, Ukrainian, Belorussian,
% Rusyn, Serbian, Macedonian and Bulgarian alphabets).

%%% I
%%% Cyrillic letters shared by all Slavic languages:

\def\cyr{}

\DeclareUTFcharacter[\UTFencname]{x0410}{\CYRA} % А
\DeclareUTFcharacter[\UTFencname]{x0430}{\cyra} % а
\DeclareUTFcharacter[\UTFencname]{x0411}{\CYRB} % Б
\DeclareUTFcharacter[\UTFencname]{x0431}{\cyrb} % б
\DeclareUTFcharacter[\UTFencname]{x0412}{\CYRV} % В 
\DeclareUTFcharacter[\UTFencname]{x0432}{\cyrv} % в
\DeclareUTFcharacter[\UTFencname]{x0413}{\CYRG} % Г
\DeclareUTFcharacter[\UTFencname]{x0433}{\cyrg} % г
\DeclareUTFcharacter[\UTFencname]{x0414}{\CYRD} % Д
\DeclareUTFcharacter[\UTFencname]{x0434}{\cyrd} % д
\DeclareUTFcharacter[\UTFencname]{x0415}{\CYRE} % Е 
\DeclareUTFcharacter[\UTFencname]{x0435}{\cyre} % е
\DeclareUTFcharacter[\UTFencname]{x0416}{\CYRZH} % Ж 
\DeclareUTFcharacter[\UTFencname]{x0436}{\cyrzh} % ж
\DeclareUTFcharacter[\UTFencname]{x0417}{\CYRZ} % З
\DeclareUTFcharacter[\UTFencname]{x0437}{\cyrz} % з
\DeclareUTFcharacter[\UTFencname]{x0418}{\CYRI} % И
\DeclareUTFcharacter[\UTFencname]{x0438}{\cyri} % и
\DeclareUTFcharacter[\UTFencname]{x041A}{\CYRK} % К
\DeclareUTFcharacter[\UTFencname]{x043A}{\cyrk} % к
\DeclareUTFcharacter[\UTFencname]{x041B}{\CYRL} % Л
\DeclareUTFcharacter[\UTFencname]{x043B}{\cyrl} % л 
\DeclareUTFcharacter[\UTFencname]{x041C}{\CYRM} % М
\DeclareUTFcharacter[\UTFencname]{x043C}{\cyrm} % м
\DeclareUTFcharacter[\UTFencname]{x041D}{\CYRN} % Н
\DeclareUTFcharacter[\UTFencname]{x043D}{\cyrn} % н
\DeclareUTFcharacter[\UTFencname]{x041E}{\CYRO} % О
\DeclareUTFcharacter[\UTFencname]{x043E}{\cyro} % о
\DeclareUTFcharacter[\UTFencname]{x041F}{\CYRP} % П
\DeclareUTFcharacter[\UTFencname]{x043F}{\cyrp} % п
\DeclareUTFcharacter[\UTFencname]{x0420}{\CYRR} % Р
\DeclareUTFcharacter[\UTFencname]{x0440}{\cyrr} % р
\DeclareUTFcharacter[\UTFencname]{x0421}{\CYRS} % С
\DeclareUTFcharacter[\UTFencname]{x0441}{\cyrs} % с
\DeclareUTFcharacter[\UTFencname]{x0422}{\CYRT} % Т
\DeclareUTFcharacter[\UTFencname]{x0442}{\cyrt} % т
\DeclareUTFcharacter[\UTFencname]{x0423}{\CYRU} % У
\DeclareUTFcharacter[\UTFencname]{x0443}{\cyru} % у
\DeclareUTFcharacter[\UTFencname]{x0424}{\CYRF} % Ф
\DeclareUTFcharacter[\UTFencname]{x0444}{\cyrf} % ф
\DeclareUTFcharacter[\UTFencname]{x0425}{\CYRH} % Х
\DeclareUTFcharacter[\UTFencname]{x0445}{\cyrh} % х
\DeclareUTFcharacter[\UTFencname]{x0426}{\CYRC} % Ц
\DeclareUTFcharacter[\UTFencname]{x0446}{\cyrc} % ц
\DeclareUTFcharacter[\UTFencname]{x0427}{\CYRCH} % Ч
\DeclareUTFcharacter[\UTFencname]{x0447}{\cyrch} % ч
\DeclareUTFcharacter[\UTFencname]{x0428}{\CYRSH} % Ш
\DeclareUTFcharacter[\UTFencname]{x0448}{\cyrsh} % ш

%%% II
%%% Letters specific to Eastern Slavic languages
%%% (Russian, Ukrainian, Belorussian, Rusyn) and for
%%% Bulgarian:
\DeclareUTFcharacter[\UTFencname]{x0401}{\CYRYO} % Ё
\DeclareUTFcharacter[\UTFencname]{x0451}{\cyryo} % ё
\DeclareUTFcharacter[\UTFencname]{x0419}{\CYRISHRT} % Й
\DeclareUTFcharacter[\UTFencname]{x0439}{\cyrishrt} % й
\DeclareUTFcharacter[\UTFencname]{x042C}{\CYRSFTSN} % Ь
\DeclareUTFcharacter[\UTFencname]{x044C}{\cyrsftsn} % ь
\DeclareUTFcharacter[\UTFencname]{x042E}{\CYRYU} % Ю
\DeclareUTFcharacter[\UTFencname]{x044E}{\cyryu} % ю
\DeclareUTFcharacter[\UTFencname]{x042F}{\CYRYA} % Я
\DeclareUTFcharacter[\UTFencname]{x044F}{\cyrya} % я
  % Belorussian only:
  \DeclareUTFcharacter[\UTFencname]{x040E}{\CYRUSHRT} % Ў
  \DeclareUTFcharacter[\UTFencname]{x045E}{\cyrushrt} % ў
  % Ukrainian and Rusyn only:
  \DeclareUTFcharacter[\UTFencname]{x0490}{\CYRGUP} % Ґ
  \DeclareUTFcharacter[\UTFencname]{x0491}{\cyrgup} % ґ
  \DeclareUTFcharacter[\UTFencname]{x0407}{\CYRII} % Ї
  \DeclareUTFcharacter[\UTFencname]{x0457}{\cyrii} % ї
  \DeclareUTFcharacter[\UTFencname]{x0404}{\CYRIE} % Є
  \DeclareUTFcharacter[\UTFencname]{x0454}{\cyrie} % є 
  % Russian-Belorussian-Rusyn:
  \DeclareUTFcharacter[\UTFencname]{x042B}{\CYRERY} % Ы
  \DeclareUTFcharacter[\UTFencname]{x044B}{\cyrery} % ы
  % Russian-Belorussian:
  \DeclareUTFcharacter[\UTFencname]{x042D}{\CYREREV} % Э
  \DeclareUTFcharacter[\UTFencname]{x044D}{\cyrerev} % э
  % Ukrainian-Belorussian-Rusyn:
  \DeclareUTFcharacter[\UTFencname]{x0406}{\CYRYI} % І
  \DeclareUTFcharacter[\UTFencname]{x0456}{\cyryi} % і
  % Russian-Ukrainian-Bulgarian-Rusyn:
  \DeclareUTFcharacter[\UTFencname]{x0429}{\CYRSHCH} % Щ
  \DeclareUTFcharacter[\UTFencname]{x0449}{\cyrshch} % щ
  % Russian-Bulgarian-Rusyn:
  \DeclareUTFcharacter[\UTFencname]{x042A}{\CYRHRDSN} % Ъ
  \DeclareUTFcharacter[\UTFencname]{x044A}{\cyrhrdsn} % ъ

%%% III
%%% Specific South Slavic Letters (Serbian and Macedonian).

\DeclareUTFcharacter[\UTFencname]{x0408}{\CYRJE} % Ј
\DeclareUTFcharacter[\UTFencname]{x0458}{\cyrje} % ј
\DeclareUTFcharacter[\UTFencname]{x0409}{\CYRLJE} % Љ
\DeclareUTFcharacter[\UTFencname]{x0459}{\cyrlje} % љ
\DeclareUTFcharacter[\UTFencname]{x040A}{\CYRNJE} % Њ
\DeclareUTFcharacter[\UTFencname]{x045A}{\cyrnje} % њ
\DeclareUTFcharacter[\UTFencname]{x040F}{\CYRDZHE} % Џ
\DeclareUTFcharacter[\UTFencname]{x045F}{\cyrdzhe} % џ
  % Serbian only:
  \DeclareUTFcharacter[\UTFencname]{x0402}{\CYRDJE} % Ђ
  \DeclareUTFcharacter[\UTFencname]{x0452}{\cyrdje} % ђ
  \DeclareUTFcharacter[\UTFencname]{x040B}{\CYRTSHE} % Ћ
  \DeclareUTFcharacter[\UTFencname]{x045B}{\cyrtshe} % ћ
  % Macedonian only:
  \DeclareUTFcharacter[\UTFencname]{x0405}{\CYRDZE} % Ѕ
  \DeclareUTFcharacter[\UTFencname]{x0455}{\cyrdze} % ѕ
  \DeclareUTFcharacter[\UTFencname]{x0403}{\CYRGJE}  % Ѓ
  \DeclareUTFcharacter[\UTFencname]{x0453}{\cyrgje}  % ѓ
  \DeclareUTFcharacter[\UTFencname]{x040C}{\CYRKJE}  % Ќ
  \DeclareUTFcharacter[\UTFencname]{x045C}{\cyrkje}  % ќ

%%%% IV 
%% Letters which are pre-1918 reform in Russian and
%% survived a bit later in other Slavic Cyrillic.
%% The complete list would cover all older Cyrillic (and
%% additionally Glagolitic) writing systems that were in
%% in use throughout Slavic Cyrillic world and disappeared
%% after reforms that occured at various times.
\DeclareUTFcharacter[\UTFencname]{x0462}{\CYRYAT} % Ѣ
\DeclareUTFcharacter[\UTFencname]{x0463}{\cyryat} % ѣ
\DeclareUTFcharacter[\UTFencname]{x046A}{\CYRBYUS} % Ѫ
\DeclareUTFcharacter[\UTFencname]{x046B}{\cyrbyus} % ѫ
\DeclareUTFcharacter[\UTFencname]{x0474}{\CYRIZH} % Ѵ
\DeclareUTFcharacter[\UTFencname]{x0475}{\cyrizh} % ѵ
\DeclareUTFcharacter[\UTFencname]{x0472}{\CYRFITA} % Ѳ
\DeclareUTFcharacter[\UTFencname]{x0473}{\cyrfita} % ѳ

%%%%%%%%%% NON-SLAVIC CYRILLIC LETTERS - PROBABLY INCOMPLETE
\DeclareUTFcharacter[\UTFencname]{x04C0}{\CYRpalochka} % Ӏ
\DeclareUTFcharacter[\UTFencname]{x0492}{\CYRGHCRS} % Ғ 
\DeclareUTFcharacter[\UTFencname]{x0493}{\cyrghcrs} % ғ
\DeclareUTFcharacter[\UTFencname]{x04BA}{\CYRSHHA} % Һ
\DeclareUTFcharacter[\UTFencname]{x04BB}{\cyrshha} % һ
\DeclareUTFcharacter[\UTFencname]{x0496}{\CYRZHDSC} % Җ
\DeclareUTFcharacter[\UTFencname]{x0497}{\cyrzhdsc} % җ
\DeclareUTFcharacter[\UTFencname]{x0498}{\CYRZDSC} % Ҙ
\DeclareUTFcharacter[\UTFencname]{x0499}{\cyrzdsc} % ҙ
\DeclareUTFcharacter[\UTFencname]{x049A}{\CYRKDSC} % Қ
\DeclareUTFcharacter[\UTFencname]{x049B}{\cyrkdsc} % қ
\DeclareUTFcharacter[\UTFencname]{x04A0}{\CYRKBEAK} % Ҡ
\DeclareUTFcharacter[\UTFencname]{x04A1}{\cyrkbeak} % ҡ
\DeclareUTFcharacter[\UTFencname]{x04A2}{\CYRNDSC} % Ң
\DeclareUTFcharacter[\UTFencname]{x04A3}{\cyrndsc} % ң
\DeclareUTFcharacter[\UTFencname]{x04A4}{\CYRNG} % Ҥ
\DeclareUTFcharacter[\UTFencname]{x04A5}{\cyrng} % ҥ
\DeclareUTFcharacter[\UTFencname]{x04E8}{\CYROTLD} % Ө
\DeclareUTFcharacter[\UTFencname]{x04E9}{\cyrotld} % ө
\DeclareUTFcharacter[\UTFencname]{x04AA}{\CYRSDSC} % Ҫ
\DeclareUTFcharacter[\UTFencname]{x04AB}{\cyrsdsc} % ҫ
\DeclareUTFcharacter[\UTFencname]{x04AE}{\CYRY} % Ү
\DeclareUTFcharacter[\UTFencname]{x04AF}{\cyry} % ү
\DeclareUTFcharacter[\UTFencname]{x04B0}{\CYRYHCRS} % Ұ
\DeclareUTFcharacter[\UTFencname]{x04B1}{\cyryhcrs} % ұ
\DeclareUTFcharacter[\UTFencname]{x04B2}{\CYRHDSC} % Ҳ
\DeclareUTFcharacter[\UTFencname]{x04B3}{\cyrhdsc} % ҳ
\DeclareUTFcharacter[\UTFencname]{x04B6}{\CYRCHRDSC} % Ҷ
\DeclareUTFcharacter[\UTFencname]{x04B7}{\cyrchrdsc} % ҷ
\DeclareUTFcharacter[\UTFencname]{x04D8}{\CYRSCHWA} % Ә
\DeclareUTFcharacter[\UTFencname]{x04D9}{\cyrschwa} % ә
\DeclareUTFcharacter[\UTFencname]{x04D4}{\CYRAE} % Ӕ
\DeclareUTFcharacter[\UTFencname]{x04D5}{\cyrae} % ӕ
\DeclareUTFcharacter[\UTFencname]{x049C}{\CYRKVCRS} % Ҝ
\DeclareUTFcharacter[\UTFencname]{x049D}{\cyrkvcrs} % ҝ
\DeclareUTFcharacter[\UTFencname]{x04B8}{\CYRCHVCRS} % Ҹ
\DeclareUTFcharacter[\UTFencname]{x04B9}{\cyrchvcrs} % ҹ 
\DeclareUTFcharacter[\UTFencname]{x04C3}{\CYRKHK} % Ӄ
\DeclareUTFcharacter[\UTFencname]{x04C4}{\cyrkhk} % ӄ
\DeclareUTFcharacter[\UTFencname]{x04C7}{\CYRNHK} % Ӈ
\DeclareUTFcharacter[\UTFencname]{x04C8}{\cyrnhk} % ӈ
\DeclareUTFcharacter[\UTFencname]{x04E0}{\CYRABHDZE} % Ӡ
\DeclareUTFcharacter[\UTFencname]{x04E1}{\cyrabhdze} % ӡ
\DeclareUTFcharacter[\UTFencname]{x04BC}{\CYRABHCH} % Ҽ
\DeclareUTFcharacter[\UTFencname]{x04BD}{\cyrabhch} % ҽ 
\DeclareUTFcharacter[\UTFencname]{x04BE}{\CYRABHCHDSC} % Ҿ
\DeclareUTFcharacter[\UTFencname]{x04BF}{\cyrabhchdsc} % ҿ 
\DeclareUTFcharacter[\UTFencname]{x049E}{\CYRKHCRS} % Ҟ
\DeclareUTFcharacter[\UTFencname]{x049F}{\cyrkhcrs} % ҟ  
\DeclareUTFcharacter[\UTFencname]{x0494}{\CYRGHK} % Ҕ
\DeclareUTFcharacter[\UTFencname]{x0495}{\cyrghk} % ҕ
\DeclareUTFcharacter[\UTFencname]{x04AC}{\CYRTDSC} % Ҭ
\DeclareUTFcharacter[\UTFencname]{x04AD}{\cyrtdsc} % ҭ
\DeclareUTFcharacter[\UTFencname]{x04B4}{\CYRTETSE} % Ҵ
\DeclareUTFcharacter[\UTFencname]{x04B5}{\cyrtetse} % ҵ  
\DeclareUTFcharacter[\UTFencname]{x04A6}{\CYRPHK} % Ҧ
\DeclareUTFcharacter[\UTFencname]{x04A7}{\cyrphk} % ҧ
\DeclareUTFcharacter[\UTFencname]{x04A8}{\CYRABHHA} % Ҩ
\DeclareUTFcharacter[\UTFencname]{x04A9}{\cyrabhha} % ҩ
\DeclareUTFcharacter[\UTFencname]{x04CB}{\CYRCHLDSC} % Ӌ
\DeclareUTFcharacter[\UTFencname]{x04CC}{\cyrchldsc} % ӌ

% These two letters are part of proposed Unicode for Kurdish
% and their codepoints may or may not change upon inclusion.
\DeclareUTFcharacter[\UTFencname]{x051A}{\CYRQ}
\DeclareUTFcharacter[\UTFencname]{x051B}{\cyrq}
\DeclareUTFcharacter[\UTFencname]{x051C}{\CYRW}
\DeclareUTFcharacter[\UTFencname]{x051D}{\cyrw}

%% These are probably improperly named in LaTeX
\DeclareUTFcharacter[\UTFencname]{x04CD}{\CYRMDSC} % Ӎ
\DeclareUTFcharacter[\UTFencname]{x04CE}{\cyrmdsc} % ӎ
\DeclareUTFcharacter[\UTFencname]{x04C5}{\CYRLDSC} % Ӆ 
\DeclareUTFcharacter[\UTFencname]{x04C6}{\cyrldsc} % ӆ

%%%%%%%%%%% PUNCTUATION AND UNCLEAR CASES
%%%% Exotic (punctuation, letters...)
%%%% (this list is probably incomplete)

% A dash - Defined in cyrillic.mtx to be emdash
% and should have been emdash all along.
\DeclareUTFcharacter[\UTFencname]{x2014}{\cyrdash} 

% Exotic angle brackets.
% Might still be unprintable for you.
\DeclareUTFcharacter[\UTFencname]{x27E8}{\cyrlangle}
\DeclareUTFcharacter[\UTFencname]{x27E9}{\cyrrangle}

% This particular association is not certain but I guess 
% we'll hear about it when someone trips over it.
\DeclareUTFcharacter[\UTFencname]{x0510}{\CYREPS}
\DeclareUTFcharacter[\UTFencname]{x0511}{\cyreps}

%Can't confirm those, but I'm positive they're correct:
\DeclareUTFcharacter[\UTFencname]{x04F6}{\CYRGDSC}
\DeclareUTFcharacter[\UTFencname]{x04F7}{\cyrgdsc}
\DeclareUTFcharacter[\UTFencname]{x04FC}{\CYRHHK}
\DeclareUTFcharacter[\UTFencname]{x04FD}{\cyrhhk}
\DeclareUTFcharacter[\UTFencname]{x0512}{\CYRLHK}
\DeclareUTFcharacter[\UTFencname]{x0513}{\cyrlhk}

% I could not assign these in a way that would make sense,
% since I couldn't find matches in cyrillic unicode table.

%\DeclareUTFcharacter[\UTFencname]{x????}{\CYRNLHK}
%\DeclareUTFcharacter[\UTFencname]{x????}{\cyrnlhk}
%\DeclareUTFcharacter[\UTFencname]{x????}{\CYRRDSC}
%\DeclareUTFcharacter[\UTFencname]{x????}{\cyrrdsc}

% Oddities:
% I don't see a letter like that in Unicode cyrillic table.
% I've no idea what is it doing in a cyrillic encoding either.
%\DeclareUTFcharacter[\UTFencname]{x????}{\CYRDELTA}
%\DeclareUTFcharacter[\UTFencname]{x????}{\cyrdelta}
